\documentclass{article}
\usepackage[utf8]{inputenc}

\title{Supplementary Materials Organization}
\date{\vspace{-5ex}}

% Package includes --------------
\usepackage[left=1.5cm, right=2cm]{geometry}
\usepackage[utf8]{inputenc}
\usepackage[english]{babel}
\usepackage{enumerate}
\usepackage{mathrsfs}
\usepackage{color}
\def\cc{\color{blue}}
\usepackage{physics}

\usepackage{amsthm}
\usepackage[normalem]{ulem}
%\usepackage[labelformat=empty]{caption}
\usepackage{bm}
\usepackage{arydshln} % allows dashed lines
\usepackage{amsmath, amssymb, amsxtra, amsfonts}
\usepackage{epsfig, epsf, epic, graphicx}
\usepackage{fancybox, lscape, subfigure}
\usepackage{graphics}
\usepackage{natbib}
\usepackage{hyperref}
\usepackage{comment}
\usepackage{listings}
\usepackage[svgnames]{xcolor}
\usepackage[T1]{fontenc}

\lstset{language=R,
    basicstyle=\small\ttfamily,
    stringstyle=\color{DarkGreen},
    otherkeywords={0,1,2,3,4,5,6,7,8,9},
    morekeywords={TRUE,FALSE},
    deletekeywords={data,frame,length,as,character},
    keywordstyle=\color{blue},
    commentstyle=\color{DarkGreen},
}

% New commands defined --------
\newcommand{\newsec}[1]{\begin{center}
    \dotfill \textit{#1} \dotfill
\end{center}}

\newcommand{\bb}{\boldsymbol \beta}
\newcommand{\ba}{\boldsymbol \alpha}


\begin{document}

\noindent \textbf{\LARGE Supplementary Material F: Aalen-Johansen, Nelson-Aalen, and \texttt{prodint} applications}\\
\vspace{0.25in}

\noindent The contents include: 
\begin{enumerate}[(a)]
\item violin plots for an additional simulation study where using the Aalen-Johansen and Nelson-Aalen estimators result in unbiased estimation of transition rate parameters
\item computational time differences between using approach (A) with \texttt{deSolve} versus with \texttt{prodint}
\end{enumerate}
\vspace{0.25in}

\section*{Part (a)}

\newpage

\section*{Part (b)}
Using the same simulated data and modeling structure as found in Section 3.3 of the manuscript, we can compare the computational time differences between using approach (A) with \texttt{deSolve} as opposed to with \texttt{prodint}. In order for \texttt{prodint} to numerically compute the transition probability matrix between two time points, say $t_k$ and $t_{k+1}$ ($t_{k+1} > t_{k}$), it requires defining a step-size from which the interval between $t_k$ and $t_{k+1}$ is discretized into smaller subintervals of length equal to the step-size. Because the interval of time over which we are numerically integrating changes within and across subjects in the data, we define the step-size to take the general form 
$$\text{step-size} = \frac{t_{k+1} - t_{k}}{s}$$
where $s$ is a fixed, positive integer defining the number of discretized subintervals (e.g., if $s=2$ then the interval between $t_k$ and $t_{k+1}$ is split into two equal subintervals with step-size equal to $\frac{t_{k+1} - t_{k}}{2}$). Recall that applying approach (A) with \texttt{deSolve} does \textit{not} require defining a step-size. Thus, for a fixed data set, the likelihood is computed using approach (A) with \texttt{deSolve}. Then, using the same data set, the likelihood is computed using approach (A) with \texttt{prodint} for various step-sizes. \begin{figure}[!htb]
\centering
\includegraphics[scale =.225]{Supplement/AJ/prodint_like.png}\\
\includegraphics[scale =.225]{Supplement/AJ/prodint_time.png}
\caption{\footnotesize Differences between using approach (A) with \texttt{deSolve} compared to with \texttt{prodint}. Note that the red dotted line in the bottom figure represents the computation time for evaluating the likelihood using \texttt{deSolve}. }\label{fig:b1}
\end{figure}
Figure \ref{fig:b1} presents the squared difference in likelihood between using \texttt{prodint} over multiple values of $s$ with the likelihood computed using \texttt{deSolve} (top), as well as how the computation time using \texttt{prodint} changes as $s$ increases (bottom). We see that for a smaller number of subintervals (i.e., larger magnitude step-size), \texttt{prodint} computes a likelihood with a larger deviation away from the likelihood computed using \texttt{deSolve}. Additionally, we see that as the number of subintervals increases (i.e., step-size decreases), the computation time using \texttt{prodint} increases almost linearly. Hence, because \texttt{prodint} requires defining a step-size size, it quickly becomes computationally infeasible compared to using \texttt{deSolve}; however, aside from computation time, we see that the two numerical integration techniques lead to nearly the exact same likelihood computations, for a sufficiently large number of subintervals (i.e., small step-size) in the case of \texttt{prodint}.

\end{document}


